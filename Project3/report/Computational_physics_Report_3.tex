\documentclass[10pt,a4paper]{article}
\usepackage[utf8]{inputenc}
\usepackage{amsmath}
\newcommand{\RomanNumeralCaps}[1]
    {\MakeUppercase{\romannumeral #1}}
\usepackage{amsfonts}
\usepackage{amssymb}
\usepackage{float}
\usepackage{verbatim}
\usepackage{listings}
\usepackage{hyperref}

\usepackage[nottoc,numbib]{tocbibind}
\usepackage{cite}

\usepackage{color} %red, green, blue, yellow, cyan, magenta, black, white
\definecolor{mygreen}{RGB}{28,172,0} % color values Red, Green, Blue
\definecolor{mylilas}{RGB}{170,55,241}
\usepackage{amsmath}
\usepackage{graphicx}
\graphicspath{{C:/Users/adria/Pictures/FYS3150/}} 
\author{Adrian Martinsen Kleven, Simon Schrader}
\title{Project 3}

\lstset{
 	language =C++,   
    frame=tb, % draw a frame at the top and bottom of the code block
    tabsize=4, % tab space width
    showstringspaces=false, % don't mark spaces in strings
    numbers=left, % display line numbers on the left
    commentstyle=\color{green}, % comment color
    keywordstyle=\color{blue}, % keyword color
    stringstyle=\color{red} % string color
}

\begin{document}

\part*{-Project 3 - FYS3150/FYS4150-
}
{\large By Simon Schrader (4150), Adrian Kleven (3150) - autumn 2019
}
\tableofcontents

\listoffigures
\listoftables

 
\clearpage
 
\section{Abstract}
Determining the ground state correlation energy between two electrons in a helium atom can be done by evaluating a certain six- dimensional integral assuming that the electrons can be modelled separately, as two, single- particle wave functions of an electron in the hydrogen atom \cite{Problem_set_3}. To solve this integral, two different approaches, relying on Gaussian quadrature and Monte- Carlo integration are used.
\section{Introduction} 
The purpose of this article is to apply different versions of Gaussian quadrature and Monte- Carlo integration in solving the six- dimensional integral
\section{Methods}

\subsection{Monte Carlo Integration}
A different approach to solving integrals numerically is the use of Monte Carlo integration, where properties of probability distribution functions (PDFs) are used to approximate the solution. For any integral $\int_a^bf(x)dx$, one can find a PDF p(x) that fulfills  $\int_a^bp(x)dx=1$ that is nonzero $ \forall x\in [a,b]$ Then by the law of large numbers \cite{devore2012modern},
$$\int_a^bf(x)dx=\int_a^bp(x)\frac{f(x)}{p(x)}dx=E[\frac{f(x)}{p(x)}]\approx \frac{1}{N}\sum_{i=1}^{N}\frac{f(x_i)}{p(x_i)}$$
where N is a very large number and $x_i$ are random samples from the given PDF.\\
The variance is then given by
\[ 
\sigma^2=\frac{1}{N}\sum_{i=1}^{N}\left(\left( \frac{f(x_i)}{p(x_i)}\right)^2-E[\frac{f(x)}{p(x)}]^2\right) p(x_i)=E[ ( \frac{f(x)}{p(x)})^2]-E[\frac{f(x)}{p(x)}]^2
\]
It can then be shown \cite{devore2012modern} that the standard error of the mean is
$$
\sigma_N \approx \frac{\sigma}{\sqrt{N}}
$$
Thus, the error is a function of $\frac{1}{\sqrt{N}}$. This is does not depend on the dimensionality of the integral to be evaluated, making Monte Carlo methods very effective for integrals in higher dimensions.
The appropriate choice of a fitting PDF is crucial. Even though the above equations hold true for any PDF, choosing a PDF p(x) that closely follows our function f(x), leads to a better sampling of $x_i$ \cite{devore2012modern} and decreases the standard deviation. Thus, the real integral is approached much faster.

\subsection{Implementation}

\subsubsection{Legendre polynomials}
The first approach to solving the integral is to use Gauss-Legendre polynomials. As Legendre polynomials cannot be properly mapped to $(-\infty,\infty)$, it is necessary to define a threshold $\lambda$ where the function is "sufficiently" zero. Thus, the integral is only evaluated for $(-\lambda,\lambda)$ As can be seen in FIGURE XXX, the function approaches zero rather quickly, and $\lambda$=2 seems to be an appropriate choice. The integration limit is then changed to $[-2,2]$ for all 6 integrands. Numerical recipe's gauleg-function \cite{press1992numerical} then maps the weights from $[-1,1]$ to $[-2,2]$. Because any all the 6 variables in the integral are freely interchangeable, the the approximation then reads, 
$$\int_{-\infty}^{\infty}\int_{-\infty}^{\infty}\int_{-\infty}^{\infty}\int_{-\infty}^{\infty}\int_{-\infty}^{\infty}\int_{-\infty}^{\infty}f(x_1,x_2,y_1,y_2,z_1,z_2)dx_1dx_2dy_1dy_2dz_1dz_2$$
$$\approx\int_{-2}^{2}\int_{-2}^{2}\int_{-2}^{2}\int_{-2}^{2}\int_{-2}^{2}\int_{-2}^{2}f(x_1,x_2,y_1,y_2,z_1,z_2)dx_1dx_2dy_1dy_2dz_1dz_2 $$
$$\approx\sum_{i=1}^{N}\sum_{j=1}^{N}\sum_{k=1}^{N}\sum_{l=1}^{N}\sum_{m=1}^{N}\sum_{n=1}^{N}\omega_i\omega_j\omega_k\omega_l\omega_m\omega_nf(x_i,x_j,x_k,x_l,x_m,x_n)$$
where $x_i$ and $\omega_i$ where created using the gauleg-function. 
Computationally, this will lead to a total of $N^6$ function evaluations. A possible problem to face is that one will end up dividing by zero $N^3$ times. This problem can be faced by ignoring all function evaluations where the function's denominator is larger than a threshold, which was chosen to be $10^{-8}.$
\subsubsection{Laguerre and Legendre polynomials}
When transferred into spherical coordinates, the integral reads 

$$
\int_{0}^{\pi}\int_{0}^{\pi}\int_{0}^{2\pi}\int_{0}^{2\pi}\int_{0}^{\infty}\int_{0}^{\infty}
\frac{r_1^2r_2^2sin(\theta_1)sin(\theta_2)e^{-4(r_1+r_2)}}{\sqrt{r_1^2+r_2^2-2r_1r_2cos(\beta)}}dr_1dr_2d\phi_1d\phi_2d\theta_1d\theta_2
$$
with
$$
cos(\beta)=cos(\theta_1)cos(\theta_2)+sin(\theta_1)sin(\theta_2)cos(\phi_1-\phi_2)
$$

This has the advantage that infinity only has to be faced twice, and Laguerre-Polynomials can be used for that. Laguerre-polynomials are defined for $[0,\infty)$ and have a weight function $W(x)=e^{-x}$. This is relevant for $r_1$ and $r_2$, as the integration limits go from $0$ to $\infty$. The angles $\theta_1$ and $\theta_2$ lie in $[0,\pi]$, and the angles $\phi_1$ and $\phi_2$  lie in $[0,2\pi]$. Here, Lagrange-Polynomials can be used again. Because each of the $\theta$, $\phi$ and $r$ are interchangeable, it is necessary to create 3 types of mesh points and weights: One using the Laguerre polynomials  ($\omega_r$, $x_r$), one using Lagrange polynomials for the angle $\theta$ ($\omega_\theta$, $x_\theta$) and one using Lagrange polynomials for the angle $\phi$ ($\omega_\phi$, $x_\phi$). The function to be evaluated changes slightly due to the Laguerre-polynomials weight function, the exponent goes from -4 to -3, leading to
$$g(r_1,r_2,\theta_1,\theta_2,\phi_1,\phi_2)=\frac{r_1^2r_2^2sin(\theta_1)sin(\theta_2)e^{-3(r_1+r_2)}}{\sqrt{r_1^2+r_2^2-2r_1r_2cos(\beta)}}$$
$$\int_{0}^{\pi}\int_{0}^{\pi}\int_{0}^{2\pi}\int_{0}^{2\pi}\int_{0}^{\infty}\int_{0}^{\infty}
g(r_1,r_2,\theta_1,\theta_2,\phi_1,\phi_2)dr_1dr_2d\phi_1d\phi_2d\theta_1d\theta_2$$
This integral is then approximated by the following sum:

$$
\approx \sum_{i=1}^N \sum_{j=1}^N \sum_{k=1}^N \sum_{l=1}^N \sum_{m=1}^N \sum_{n=1}^N \omega_{r,i} \omega_{r,j} \omega_{\theta ,k} \omega_{\theta ,l} \omega_{\phi ,m} \omega_{\phi ,n} g(x_{r,i},x_{r,j},x_{\theta ,k},x_{\theta ,l},x_{\phi ,m},x_{\phi ,n})
$$

\section{Results}

\section{Conclusion}

\section{Critique}

\section{Appendix}


\subsection{List of programs}
All programs can be found on \url{https://github.com/adrian2208/FYS3150_collab} in the folder "project 3".
\begin{itemize}
\item[1.] 
\end{itemize}

\subsection{Tables}

\cite{Lecture_Notes_Fall_2015}
\cite{Problem_set_3}
\bibliographystyle{Plain}



\bibliography{citations}



\begin{comment}

$$
\begin{bmatrix}
0 & 0 & 0 & 0 \\
0 & 0 & 0 & 0 \\
0 & 0 & 0 & 0 \\
0 & 0 & 0 & 0 \\
\end{bmatrix}
$$

\begin{lstlisting}[caption=insert caption]
for (unsigned int i = 0; i<100;i++{
}
\end{lstlisting}

\begin{figure}[h]
\includegraphics[width=8cm]{}
\caption{include caption}
\end{figure}

\end{comment}

\end{document}