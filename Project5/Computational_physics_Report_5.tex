\documentclass[10pt,a4paper]{article}
\usepackage[utf8]{inputenc}
\usepackage{amsmath}
\newcommand{\RomanNumeralCaps}[1]
    {\MakeUppercase{\romannumeral #1}}
\usepackage{amsfonts}
\usepackage{amssymb}
\usepackage{float}
\usepackage{verbatim}
\usepackage{listings}
\usepackage{hyperref}
\usepackage{pbox}
\usepackage{graphicx}
%bibliography packages, bibliography files are plain text files marked .bib
%... in the same directory as the .tex file.
\usepackage[nottoc,numbib]
{tocbibind}
\usepackage{cite}
\usepackage{braket}
\usepackage{color} %red, green, blue, yellow, cyan, magenta, black, white
\definecolor{mygreen}{RGB}{28,172,0} % color values Red, Green, Blue
\definecolor{mylilas}{RGB}{170,55,241}
\usepackage{amsmath}
\usepackage{graphicx}
\graphicspath{{C:/Users/adria/Pictures/FYS3150/}}
\author{Adrian Martinsen Kleven, Simon Schrader}
\title{Project 5}

\lstset{
 	language =C++,
    frame=tb, % draw a frame at the top and bottom of the code block
    tabsize=4, % tab space width
    showstringspaces=false, % don't mark spaces in strings
    numbers=left, % display line numbers on the left
    commentstyle=\color{green}, % comment color
    keywordstyle=\color{blue}, % keyword color
    stringstyle=\color{red} % string color
}
\setlength{\columnsep}{10mm}
%\setlength{\tabcolsep}{18pt}
%\renewcommand{\arraystretch}{1.5}

\begin{document}

\part*{-Project 5 - FYS3150/FYS4150-
}
{\large \textbf{Quantum Monte Carlo of confined electrons}}\\
{\large By Simon Schrader (4150), Adrian Kleven (3150) - autumn 2019
}
\tableofcontents

\listoffigures
\listoftables


\clearpage

\section{Abstract}


\section{Introduction}
The purpose of this article is to apply the Variational Monte Carlo (VMC) method to two electrons in a quantum dot, modelled by a 3- dimensional harmonic oscillator potential. This is in order to evaluate their ground state energy, relative distance, and the expectation values of their kinetic and potential energies.\\In order to benchmark the results of this method, we are comparing the results of the ground state energy with analytical solutions for specific harmonic oscillator frequencies (henceforth $\omega$) as found in M. Taut's \cite{PhysRevA.48.3561} and our previous article \cite{Project2} on the topic of electron confinement in 3- dimensional harmonic oscillator potentials.\\\\The idealized Hamiltonian used in this article, models the Coulombic interaction between electrons and the 3- dimensional harmonic oscillator potential: 
\begin{equation}
  \label{eq:finalH}
  \hat{H}=\sum_{i=1}^{N} \left(  -\frac{1}{2} \nabla_i^2 + \frac{1}{2} \omega^2r_i^2  \right)+\sum_{i<j}\frac{1}{r_{ij}}
\end{equation}
\cite{Problem_set_5} with $r_{ij}$ being the distance between electrons $i$ and $j$ with position moduli $r_i$ and $r_j$ respectively. The natural units ($\hbar = c = e = m_e =1$) are used, and energy is given in atomic units. \\\\The VMC method is implemented using C++ (see list of programs: \ref{Listofprograms}).
\section{Methods}
\subsection{VMC}
\subsection{Non- interacting system}
The Hamiltonian for two non- interacting electrons in a harmonic oscillator potential is
\begin{equation}
\hat{H}_{0} = \frac{1}{2}\left( \omega^2   \left( r_1^2+r_2^2 \right)-\left( \nabla_1^2 + \nabla_2^2 \right)  \right)
\end{equation}
with an exact energy of 3 atomic units (a.u) \cite{Problem_set_5}.\\The unperturbed wave function for the ground state of a two- electron system is 
\begin{equation*}
\Phi(\mathbf{r}_1,\mathbf{r}_2) = C\exp{\left(-\omega(r_1^2+r_2^2)/2\right)}
\end{equation*}
\cite{Problem_set_5}.\\This wave function is symmetric under the interchange of electron labels, meaning
\begin{equation*}
\hat{I}\Phi(\mathbf{r}_1,\mathbf{r}_2) = \Phi(\mathbf{r}_1,\mathbf{r}_2)
\end{equation*}
where $\hat{I}$ is the label- interchange operator. Electrons, being fermions are required to be anti- symmetric under the label- interchange operator which implies that their spins are necessarily oppositely oriented as to fulfil the requirement of anti- symmetry \cite{griffiths2018introduction}.
\subsection{The variational principle}
The variational principle in quantum mechanics gives rise to a method by which you can get an upper bound for the ground state energy of a (time- independent) Hamiltonian which you are otherwise unable to solve.\\ 
By choosing any normalizable trial wave function $\ket{\psi}$, the expectation value of the Hamiltonian is certain to be equal to or greater than the ground state energy ($E_{gs}$) of the system.
$$
\braket{H}\equiv \frac{\braket{\psi|\hat{H}|\psi}}{\braket{\psi|\psi}} \geq E_{gs}
$$
(see \ref{Proof_variational_principle}). We then have a procedure for estimating $E_{gs}$:
\begin{enumerate}
\item Pick/guess a class of states $\ket{\psi_\beta}$ that depends on one or more (variational) parameters $\beta$.
\item calculate $E_{trial}=\frac{\braket{\psi_\beta|\hat{H}|\psi_\beta}}{\braket{\psi_\beta|\psi_\beta}}$
\item minimize $E_{trial}$ with respect to $\beta$ (and any other variational parameters). This can be done computationally or by finding the zeros of the derivative.
\end{enumerate}
$E_{trial}$ is then an upper bound on $E_{gs}$. The variational method gives the lowest upper bound within the class of trial wave functions. It is therefore of greatest importance to choose an appropriate trial wave function in order to minimize the energy discrepancy.
\paragraph{The trial wave functions} used in this model are \begin{equation}\label{1sttrialwavefunction}
\Psi_{T1}(\mathbf{r}_1,\mathbf{r}_2) = C\exp{\left(-\alpha\omega(r_1^2+r_2^2)/2\right)}
\end{equation}
and
\begin{equation}\label{2ndtrialwavefunction}
\Psi_{T2}(\mathbf{r}_1,\mathbf{r}_2) =
    C\exp{\left(-\alpha\omega(r_1^2+r_2^2)/2\right)}
    \exp{\left(\frac{r_{12}}{2(1+\beta r_{12})}\right)}
\end{equation}
with variational parameters $\alpha$ and $\beta$. 
\subsubsection{The cusp condition}
We examine the behaviour of the trial wave functions \eqref{1sttrialwavefunction} and \eqref{2ndtrialwavefunction} when $r_{12}\rightarrow 0$:
\begin{equation}
\Psi_{T1}(\mathbf{r}_1,\mathbf{r}_2)_{r_{12}\rightarrow 0} = C\exp{\left(-\alpha\omega r_1^2\right)}
\end{equation}
\begin{equation}
\Psi_{T2}(\mathbf{r}_1,\mathbf{r}_2)_{r_{12}\rightarrow 0} = C\exp{\left(-\alpha\omega r_1^2\right)}
\end{equation}
\paragraph{The total spin of the system is zero}

\subsection{Values of interest}

The Hamiltonian for a two- electron system with electron- electron repulsion is
\begin{equation}
\hat{H}_{N=2} = \frac{1}{2}\left( \omega^2   \left( r_1^2+r_2^2 \right)-\left( \nabla_1^2 + \nabla_2^2 \right)  \right)+\frac{1}{r_{12}}
\end{equation}
With kinetic energy 
\begin{equation}
\hat{T} = - \frac{1}{2}\left( \nabla_1^2 + \nabla_2^2 \right) 
\end{equation}
and potential energy 
\begin{equation}
\hat{V}=  \frac{1}{2} \omega^2   \left( r_1^2+r_2^2 \right) +\frac{1}{r_{12}}.
\end{equation}
We are interested in obtaining the expectation value of the energy of the system:
\begin{equation}
\langle H \rangle = \frac{\int d{\bf
R}\Psi^{\ast}_T({\bf R})\hat{H}({\bf R})\Psi_T({\bf R})}
{\int d{\bf
R}\Psi^{\ast}_T({\bf R})\Psi_T({\bf R})}
\end{equation}
which, since $\Psi_T({\bf R})$ are eigenfunctions of $\hat{H}$, can be written as 
\begin{equation}\label{Energy_expectation_value}
\langle H \rangle =\int P({\bf R})E_L({\bf R})d{\bf R},
\end{equation}
Where $E_L({\bf R})$ is the local energy of the system, and $P({\bf R})$ is the probability distribution function of the wave function.
\paragraph{The Virial theorem}states that the expectation value of the total kinetic energy is proportional to the expectation value of the total potential energy, and that for a pure harmonic oscillator, the constant of proportionality is 1 \cite{Problem_set_5}. We wish to examine whether this is true of our system, and so we need expressions for the total potential- and kinetic energy.\\Thanks to the linearity of the expectation value operator, we have 
\begin{equation}
\hat{H}=\hat{T}+\hat{V},
\end{equation}
\begin{equation}
\braket{H} = \braket{T} + \braket{V}.
\end{equation}
$\hat{V}$ is a linear operator. Acting on a wave function then, it give the same relation as \eqref{Energy_expectation_value}:
\begin{equation}
\braket{V} = \int P({\bf R})V({\bf R})d{\bf R}.
\end{equation}
We wish to have two different expressions for $\braket{T}$ and $\braket{V}$ depending on whether we include electron repulsion or have a pure harmonic oscillator. We get
\begin{equation}
\braket{V} = \int P({\bf R})\left( \frac{1}{2} \omega^2   \left( r_1^2+r_2^2 \right) +\frac{1}{r_{12}} \right)d{\bf R}
\end{equation}
when including electron repulsion and 
\begin{equation}
\braket{V} = \int P({\bf R})\left( \frac{1}{2} \omega^2   \left( r_1^2+r_2^2 \right) \right)d{\bf R}
\end{equation}
for the pure harmonic oscillator potential.\\$\hat{T}$, unlike $\hat{V}$ gives different values for its corresponding observable value depending on the wave function it acts on and its computation is a little more involved, and so we will obtain $\braket{T}$ simply, by the relation
\begin{equation}
\braket{T} = \braket{H}-\braket{V}.
\end{equation}

\paragraph{To find $E_L$}, as required in \eqref{Energy_expectation_value}, we apply $\hat{H}_{N=2}$ to its eigenfunctions:
\begin{equation*}
\hat{H}_{N=2}\Psi_{T1} = \left( \frac{1}{2}\left( \omega^2   \left( r_1^2+r_2^2 \right)-\left( \nabla_1^2 + \nabla_2^2 \right)  \right)+\frac{1}{r_{12}} \right)C\exp{\left(-\alpha\omega(r_1^2+r_2^2)/2\right)}
\end{equation*}
The radially dependent part of the Laplace operator being
\begin{equation}
\nabla_i^2 = \frac{1}{r_i^2}\frac{d}{dr_i}\left( r_i^2 \frac{d}{dr_i} \right).
\end{equation}
The Laplacian $\nabla_i^2$ with respect to $r_i$ acting on the wave function $\Psi_{T1}$ gives
\begin{equation}
\nabla_i^2Ce^{\left(-\alpha\omega(r_1^2+r_2^2)/2\right)} = -\alpha\omega \left( 3 -\alpha \omega r_i^2 \right)Ce^{\left(-\alpha\omega(r_1^2+r_2^2)/2\right)}
\end{equation}
so that 
\begin{equation}
\nabla_i^2\Psi_{T1} = -\alpha\omega \left( 3 -\alpha \omega r_i^2 \right)\Psi_{T1}.
\end{equation}
Applying this for both $r_1$ and $r_2$, then gives
\begin{equation}
\left( \nabla_1^2 + \nabla_2^2 \right)\Psi_{T1} = \left( -6\alpha \omega+\alpha^2\omega^2\left( r_1^2+r_2^2 \right) \right)\Psi_{T1}
\end{equation}
while 
\begin{equation}
\hat{V}\Psi_{T1} = \left( \frac{1}{2} \omega^2   \left( r_1^2+r_2^2 \right) +\frac{1}{r_{12}} \right)\Psi_{T1}.
\end{equation}
We get
\begin{equation}
\hat{H}_{N=2}\Psi_{T1} = \left( \frac{1}{2}\omega^2\left( r_1^2+r_2^2\right)\left(1-\alpha^2\right) +3\alpha\omega+\frac{1}{r_{12}} \right)\Psi_{T1}
\end{equation}
which means the local energy expression is given by
\begin{equation}
E_{L1} =  \frac{1}{2}\omega^2\left( r_1^2+r_2^2\right)\left(1-\alpha^2\right) +3\alpha\omega+\frac{1}{r_{12}}.
\end{equation}
Similarly, when applied to the second trial wave function,
\begin{equation*}
\hat{H}_{N=2}\Psi_{T2} = \left( E_{L1}+\frac{1}{2(1+\beta r_{12})^2}\left\{\alpha\omega r_{12}-\frac{1}{2(1+\beta r_{12})^2}-\frac{2}{r_{12}}+\frac{2\beta}{1+\beta r_{12}}\right\} \right)\Psi_{T2}
\end{equation*}
and so 
\begin{equation}
E_{L2} = E_{L1}+\frac{1}{2(1+\beta r_{12})^2}\left\{\alpha\omega r_{12}-\frac{1}{2(1+\beta r_{12})^2}-\frac{2}{r_{12}}+\frac{2\beta}{1+\beta r_{12}}\right\}
\end{equation}
\cite{Problem_set_5}.
\section{Results}

\section{Conclusion}

\section{Critique}

\section{Appendix}
\subsection{Proof of the variational principle in quantum mechanics}\label{Proof_variational_principle}
Given a normalized trial wave function $\ket{\psi}$. As any normalizable wave function can be expressed as a linear combination of energy eigenstates, we have that
$$
\ket{\psi} = \sum\limits_n C_n \ket{E_n}
$$
where $C_n$ is a complex coefficient, $\ket{E_n}$ is the energy eigenstate corresponding to the energy $E_n$ and $E_0=E_{gs}$. Now, calculating $\braket{H}$, 
$$
\braket{\psi|\hat{H}|\psi} = \sum\limits_{nm} \braket{E_m|C_m^*\hat{H}C_n|E_n}
$$
we get that 
$$
\braket{H}= \sum\limits_{nm} C_m^*C_n E_n \delta_{mn} =\sum\limits_{n} \big|  C_n \big|^2E_n
$$
where $\delta_{mn}$ is the Kronecker- delta.
\begin{equation}
\braket{H}=\sum\limits_{n} \big|  C_n \big|^2E_n \geq E_{gs}
\end{equation}
\subsection{Derivation of $E_{L2}$}
\begin{equation}
\nabla_i^2\Psi_{T2} = \nabla_i^2\left( \Psi_{T1}\exp{\left(\frac{r_{12}}{2(1+\beta r_{12})}\right)} \right)
\end{equation}
\[
\frac{d}{dr_i}\left(  \Psi_{T1}\exp{\left(\frac{r_{12}}{2(1+\beta r_{12})}\right)} \right) = \exp{\left(\frac{r_{12}}{2(1+\beta r_{12})}\right)}\frac{d}{dr_i}\Psi_{T1} + \Psi_{T1}\frac{d}{dr_i}\left( \exp{\left(\frac{r_{12}}{2(1+\beta r_{12})}\right)}\right)
\]
\begin{equation*}
\exp{\left(\frac{r_{12}}{2(1+\beta r_{12})}\right)}\frac{d}{dr_i}\Psi_{T1} = -\alpha \omega r_i\Psi_{T2}
\end{equation*}

\begin{equation*}
 \Psi_{T1}\frac{d}{dr_1}\left( \exp{\left(\frac{r_{12}}{2(1+\beta r_{12})}\right)}\right) = \frac{1}{4r_{12}\left( 1+\beta r_{12} \right)^2}  \Psi_{T1}\exp{\left(\frac{r_{12}}{2(1+\beta r_{12})}\right)} =  \frac{1}{4r_{12}\left( 1+\beta r_{12} \right)^2}\Psi_{T2}
\end{equation*}

\[
\frac{d}{dr_1}\Psi_{T2} = \left( \frac{1}{4r_{12}\left( 1+\beta r_{12} \right)^2} -\alpha \omega r_1\right)\Psi_{T2}
\]

\[
\frac{d}{dr_2}\Psi_{T2} = -\left( \frac{1}{4r_{12}\left( 1+\beta r_{12} \right)^2} +\alpha \omega r_2\right)\Psi_{T2}
\]

\[
\frac{d}{dr_1}\left( r_1^2 \frac{d}{dr_1}\Psi_{T2}\right) =  \frac{d}{dr_1}\left( r_1^2\left( \frac{1}{4r_{12}\left( 1+\beta r_{12} \right)^2} -\alpha \omega r_1\right)\Psi_{T2} \right)
\]

\[
= r_1^2\left( \frac{1}{4r_{12}\left( 1+\beta r_{12} \right)^2} -\alpha \omega r_1\right)^2\Psi_{T2}+\Psi_{T2} \frac{d}{dr_1}\left( r_1^2\left( \frac{1}{4r_{12}\left( 1+\beta r_{12} \right)^2} -\alpha \omega r_1\right)\right).
\]
\[
\frac{d}{dr_1}\left( r_1^2\left( \frac{1}{4r_{12}\left( 1+\beta r_{12} \right)^2} -\alpha \omega r_1\right)\right)=
\]
\[
\dfrac{r_1}{2r_{12}\left(\beta r_{12}+1\right)^2}-\dfrac{r_1^2}{8r_{12}^3\left(\beta r_{12}+1\right)^2}-\dfrac{\beta r_1^2}{4r_{12}^2\left(\beta r_{12}+1\right)^3}-3\alpha \omega r_1^2
\]

\[
\nabla_1^2\Psi_{T2} = \left( \left( \frac{1}{4r_{12}\left( 1+\beta r_{12} \right)^2} -\alpha \omega r_1\right)^2 +\dfrac{1}{2r_1r_{12}\left(\beta r_{12}+1\right)^2}-\dfrac{1}{8r_{12}^3\left(\beta r_{12}+1\right)^2}-\dfrac{\beta}{4r_{12}^2\left(\beta r_{12}+1\right)^3}-3\alpha \omega \right)\Psi_{T2}
\]
\\\\
\[
\frac{d}{dr_2}\left( r_2^2 \frac{d}{dr_2}\Psi_{T2}\right) =  \frac{d}{dr_2}\left( -r_2^2\left( \frac{1}{4r_{12}\left( 1+\beta r_{12} \right)^2} +\alpha \omega r_2\right)\Psi_{T2} \right)
\]
\[
= r_2^2\left( \frac{1}{4r_{12}\left( 1+\beta r_{12} \right)^2} +\alpha \omega r_2\right)^2\Psi_{T2} - \Psi_{T2}\frac{d}{dr_2}\left( r_2^2\left( \frac{1}{4r_{12}\left( 1+\beta r_{12} \right)^2} +\alpha \omega r_2\right) \right)
\]\\
\[
\frac{d}{dr_2}\left( r_2^2\left( \frac{1}{4r_{12}\left( 1+\beta r_{12} \right)^2} +\alpha \omega r_2\right) \right) = 
\]
\[
\dfrac{r_2}{2r_{12}\left(\beta r_{12}+1\right)^2}-\dfrac{r_2^2}{8r_{12}^3\left(\beta r_{12}+1\right)^2}-\dfrac{r_2^2\beta}{4r_{12}^2\left(\beta r_{12}+1\right)^3}+3\alpha \omega r_2^2
\]
\[
\nabla_2^2\Psi_{T2} = \left( \left( \frac{1}{4r_{12}\left( 1+\beta r_{12} \right)^2} +\alpha \omega r_2\right)^2 -   \dfrac{1}{2r_2r_{12}\left(\beta r_{12}+1\right)^2}+\dfrac{1}{8r_{12}^3\left(\beta r_{12}+1\right)^2}+\dfrac{\beta}{4r_{12}^2\left(\beta r_{12}+1\right)^3}-3\alpha \omega \right)\Psi_{T2}
\]

\[
\left( \nabla_1^2+\nabla_2^2 \right)\Psi_{T2} = \Bigg( \left( \frac{1}{4r_{12}\left( 1+\beta r_{12} \right)^2} +\alpha \omega r_2\right)^2 +\left( \frac{1}{4r_{12}\left( 1+\beta r_{12} \right)^2} -\alpha \omega r_1\right)^2- \cdots
\]
\[ 
\cdots \dfrac{1}{2r_{12}\left(\beta r_{12}+1\right)^2}\left( \frac{1}{r_1}-\frac{1}{r_2} \right)-6\alpha \omega \Bigg)\Psi_{T2}
\]
\subsection{Figures}

\subsection{Tables}

\subsection{List of programs}\label{Listofprograms}
All programs can be found on \url{https://github.com/adrian2208/FYS3150_collab} in the folder "Project5".


\begin{itemize}
\item[1.] b.cpp - Program with several options of writing to file, different matrices etc., for a given temperature and lattice size
\item[2.] e\_not\_parallel.cpp - Like b, but can run over several lattices and temperatures, more efficient
\item[3.] e.cpp - Same as above, but parallelized with openMPI
\item[4.] vecop.hpp - Several functions
\item[5.] vecop.cpp - Functions, out of which some are used.
\item[6.] tests\_main.cpp - Unit testing.
\item[7.] a.py - Analytical solution. Not properly up to date!
\item[8.] plot\_cv.py - Creates $2\times2$ plot for the larger simulations, also calculates $T_C(L_{\rightarrow \infty})$
\item[9.] plot\_distribution.py Plots the distribution histograms at different temperatures.
\item[10.] plot\_energy\_simulation\_step.py Plots the  avg. energy, the avg. magnetisation and the amount of accepted steps as a function of simulation step.
\item[11.] functions.hpp \& functions.cpp - Contains functions for parallelization, and all the integrands, and some more.
\end{itemize}

\bibliographystyle{Plain}



\bibliography{citations}



\begin{comment}

$$
\begin{bmatrix}
0 & 0 & 0 & 0 \\
0 & 0 & 0 & 0 \\
0 & 0 & 0 & 0 \\
0 & 0 & 0 & 0 \\
\end{bmatrix}
$$

\begin{lstlisting}[caption=insert caption]
for (unsigned int i = 0; i<100;i++{
}
\end{lstlisting}

\begin{figure}[h]
\includegraphics[width=8cm]{}
\caption{include caption}
\end{figure}

\end{comment}

\end{document}